\documentclass{article}

\usepackage{graphicx}
\usepackage{subcaption}
\usepackage{tikz}
\usepackage[utf8]{inputenc}
\usepackage[T1]{fontenc}
\usepackage{lmodern}
\usepackage{amsmath}
\usepackage{amsthm}
\usepackage{amsfonts}
\usepackage{amssymb}
\usepackage{enumitem}
\usepackage{commath}
\usepackage{mathtools}
\usepackage{adjustbox}
\usepackage{setspace}
\usepackage{bigints}
\usepackage{hyperref}
\usepackage{ulem}
\usepackage{esdiff}
\usepackage{pgfplots}
\usepackage{caption}
\usetikzlibrary{shapes.symbols}

\newtheorem{definicija}{Definicija}
\newtheorem{trditev}{Trditev}
\newtheorem{lema}{Lema}
\newtheorem{posledica}{Posledica}
\newtheorem{opomba}{Opomba}
\newtheorem{primer}{Primer}
\newtheorem{izrek}{Izrek}


\newcommand{\C}{\mathbb{C}}
\newcommand{\D}{\mathbb{D}}
\newcommand{\Z}{\mathbb{Z}}
\newcommand{\N}{\mathbb{N}}
\newcommand{\M}{\mathcal{M}}
\newcommand{\F}{\mathcal{F}}
\newcommand{\R}{\mathbb{R}}
\newcommand{\Ho}{\mathcal{O}}
\newcommand{\dd}{\mathrm{d}}


\title{Dinamika kompleksnih funkcij}
\author{Uroš Kosmač}

\begin{document}

\section{Kompleksna dinamika}

Kompleksno ravnino lahko kompaktificiramo z eno točko, kar nam da \textbf{Riemannovo sfero} $\hat{\C} = \C \cup \{\infty\}$.\\
Slika Riemann \\ 
Holomorfne razširitve $f$ na $\hat{C}$ (če obstaja) v okolici $z = \infty$ obravnavamo s pomočjo konjugacije s preslikavo $z\mapsto \frac{1}{z}$. Pravimo, da je $f(z)$ holomorfna v $z = \infty$, tedaj, ko je $f(\frac{1}{z})$ holomorfna v okolici $z = 0$.

\begin{primer}
\begin{enumerate}
\item[i)] $f(z) = \frac{1}{z - 1}$ $\Longrightarrow$ 
$$
f(\frac{1}{z}) = \frac{1}{\frac{1}{z} -1} = \frac{z}{1 - z} = z + z^2 + z^3 + \dots
$$
$z = 0$ je ničla prve stopnje za $f(\frac{1}{z})$, zato je $z = \infty$ ničla prve stopnje za $f$.\\ 
$f(1) = \infty$ $\Longrightarrow$
$$
\frac{1}{f(z)} = z - 1
$$
tj. $z = 1$ je ničla prve stopnje za $\frac{1}{f(z)}$, zato je tudi pol prve stopnje za $f$.
\item[ii)] $f(z) = z^2 + 1$
$$
f(\infty) = \infty \Longrightarrow \frac{1}{f(\frac{1}{z})} = \frac{1}{\frac{1}{z^2} + 1} = \frac{z^2}{1 + z^2} = z^2 + z^4 + \dots
$$
$z = \infty$ je pol 2 stopnje.
\item[iii)] $f(z) = e^z = e^x(\cos{y} + i\sin{y})$. Razširitev ni možna, saj za $x \rightarrow -\infty$ velja $|f(z)| \rightarrow 0$, ter za $x\rightarrow \infty$ velja $|f(z)| \rightarrow \infty$.
\end{enumerate}
\end{primer}

\begin{izrek}
Funkcija $f: \hat{\C} \rightarrow \hat{\C}$ je holomorfna natanko tedaj, ko je razširitev racionalne funkcije ali $f \equiv \infty$.
\end{izrek}

\begin{proof}[Dokaz:]
$(\Longleftarrow)$: sledi iz primerov, dodaj.\\ 
$(\Longrightarrow)$: 
\begin{enumerate}
\item[i)] če $f\in \Ho(\C)$ nima polov in $f(\infty) \neq \infty$ je po Liouvillovem izreku konstantna.
\item[ii)] če ima $f$ na $\C$ neskončno mnogo polov, potem zaradi diskretnosti, obstaja zaporedje le teh, ki gre proti $\infty$.\\ 
wop wop\\ 
Posledično bo tudi $f(\infty) = \infty$ zaradi zveznosti in $f \equiv \infty$ zaradi principa identičnosti.
\item[iii)] Če ima $f$ le končno mnogo polov, ima po podobnem argumentu, kot pri $ii)$ tudi kočno ničel. Potem vzamemo $R$, iracionalno funkcijo z istimi ničlami in poli. Po $i)$ je $\frac{f}{R}$ konstantna.
\end{enumerate}
Povzetek: takoj, ko ima $f$ neskočno ničel ali polov oz. kako bistveno singularnost, je nemoreš obravnavati $\hat{C}$ oz. ni holomorfna.
\end{proof}

\subsection{Fatoujeva in Juliajeva množica}

$f(z) = z^2$ oz. v polarnih koordinatah $re^{i\phi} \mapsto r^2 e^{i2\phi}$.
\begin{enumerate}
\item[$|z| > 1:$] $\lim_{n\rightarrow \infty} |f^n(z)| = \lim_{n\rightarrow \infty} r^{2n} = \infty$
\item[$|z| < 1:$] $\lim_{n\rightarrow \infty} |f^n(z)| = \lim_{n\rightarrow \infty} r^{2n} = 0$
\item[$|z| = 1:$] $f$ "deluje kot" doubling map $D$ in je kaotična. Poleg tega pa ima vsaka točka v okolici tudi točki, ki gresta proti $0$ oz. $\infty$
\end{enumerate}

Sklep: kompleksna ravnina oz. Riemannova sfera razpade na dve disjunktni množici. \textbf{Fatoujeva}, kjer je obnašanje $f^n$ "predvidljivo" in \textbf{Juliajevo}, kjer je obnašanje $f^n$ kaotično 
\begin{align*}
\mathcal{F}_f &= \hat{\C} \slash \partial \D \\ 
\mathcal{J}_f &= \partial \D.
\end{align*}
Za formalno definicijo teh dve množic rabimo koncept normalnih družin.

\begin{definicija}
Zaporedje $(f_n)_{n\in\N} \subset \Ho(D)$ za $D \subseteq \C$, konvergira k $f \in \Ho(D)$ \textbf{enakomerno po kompaktih} na $D$, če $\forall \epsilon > 0$ in $\forall K^{komp.} \subset D$ $\exists n_0\in \N$: $\forall n \geq n_0$ in $z\in K$, velja $|f_n(z) - f(z)| < \epsilon$.
\end{definicija}

\begin{primer}
$f_n(z) = z^n$ na $\D$.\\ 
Limita po točkah je enaka $f(z) = 0$. Izberemo poljuben kompakt $K\subset \D$ Zanj obstaja $r = r(K) \in [0, 1)$, da je $K \subseteq \D(0, r)$ Ker je $r < 1$, lahko naredimo oceno $\forall z\in K$:
$$
|f_n(z) - f(z)| = |z^n| \leq r^n \xrightarrow{n\rightarrow \infty} 0.
$$
tj. $\forall \epsilon > 0$ obstaja $n_0 \in \N$, da je za $n \geq n_0$: $|f_n(z) - f(z)| < \epsilon$. Seveda pa enakomerna konvergenca na celem $\D$ v tem primeru ne obstaja.
\end{primer}

\begin{izrek}
Naj bo $(f_n)_{n\in\N} \subset \Ho(D)$, $D\subseteq \C$, zaporedje ki konvergira $f_n \rightarrow f$ enakomerno po kompaktih. Potem je tudi $f \in \Ho(D)$ holomorfna.
\end{izrek}

\begin{proof}[Dokaz:]
TBA.
\end{proof}

\begin{definicija}
Družina $\F \subset \Ho(D)$, $D \subseteq \hat{C}$ je \textbf{normalna na $D$}, če za vsako zaporedje $(f_n)_{n\in \N} \subset \F$ obstaja podzaporedje, ki konvergira enakomerno po kompaktih k neki $f\in \Ho(D)$ ali k $f\equiv \infty$.
\end{definicija}

\begin{opomba}
\begin{itemize}
\item Normalne družine so analog kompaktnih množic v $\Ho(D) \cup \{f \equiv \infty\}$. 
\item Normalnost se študira tudi v drugih razredih funkcij, v katerih pa se običajno ne dodaja $f\equiv \infty$.
\item Zadostnim oz. ekvivaletnim pogojem se reče Arzela - Ascolijevi izreki. Npr. za družino $\F \subset C([a, b])$ velja, da je normalna, če je:
\begin{enumerate}
\item[i)] enakomerno omejena tj. $\exists M > 0: |f(x)| \leq M$ $\forall x\in [a, b]$ in $\forall f\in \F$.
\item[ii)] je enakomerno enakozvezna tj. $\forall \epsilon > 0$. $\exists \delta > 0: |x - y| < \delta \Longrightarrow |f(x) - f(y)| < \epsilon$ $\forall x, y\in [a, b]$ in $\forall f\in \F$.
\item[iii)] Pri holomorfnih funkcijah $ii)$ sledi iz $i)$ zaradi Cauchyjeve integracijske formule. Zaradi kompaktnosti $\hat{\C}$ je dovolj celo lokalna omejenost. Zato se Arzela-Ascolijev izrek navadn prenese na Montelova izreka:
\begin{izrek}[Prvi Montelov izrek]
$\F \subset \Ho(D)$, $D\subseteq \hat{C}$ je normalna, če je lokalno enakomerno omejena tj. $\forall z\in D$. $\exists z\in U \subset D $ in $M > 0$, da je $|f(w)| \leq M$ $\forall w\in U$ in $\forall f\in \F$.
\end{izrek}
\end{enumerate}
\end{itemize}
\end{opomba}

\begin{definicija}
Naj bo $f \in \Ho(D)$, $D\subseteq \hat{\C}$. Njena \textbf{Fatoujeva množica} $\F_f$ je definirana kot množica točk $z\in D$, za katere obstaja okolica $U\subset D$, da je družica $\{f^n \,|\, n\in \N\}$ normalna na $U$. Njena \textbf{Juliajeva množica} pa je definirana kot komplement $\mathcal{J}_f = D\slash \F_f$.
\end{definicija}

\begin{primer}
$f(z) = z^2 \Longrightarrow f^n(z) = z^{2^n}$. \\ 
Za $|z| < 1$ konvergira enakomerno po kompaktih na $\D$ $\Longrightarrow$ $\D \subset \F_f$.\\ 
Za $|z| > 1$ konvergira enakomerno po kompaktih na $\hat{C}\slash\D$ $\Longrightarrow$ $\D \subset \F_f$ k $\infty$.
Dokaz: za $K \subset \hat{C} \slash \overline{\D}$ obstaja $r > 1$, da je $\D(0, r) \cap K = \emptyset$, oz. $|z| \geq r$ za $z\in K$. \\ 
Posledično:
$$
|f^n(z)| = |z^{2^n}| \geq r^{2^n} \rightarrow \infty.
$$ 
tj. $\forall M > 0$. $\exists n_0 \in \N$, da je $|f^n(z)| > M$ za $n \geq n_0$ in $z\in K$. Alternativno bi lahko oravnavali $\frac{1}{f^n(z)} \rightarrow 0$ enakomerno po kompaktnih. \\ 
Za $|z| = 1$ normalnost nimamo v nobeni okolici, saj so blizu točke, ki gredo k $\infty$ in take, ki gredo k $0$, zato tudi, če bi obstajala limita, ne bi bila zvezna.\\ 
Sklep: ponovno smo ugotovili, da sta $\F_f = \hat{C}\slash \partial \D$ in $\mathcal{J}_f = \partial \D$.
\end{primer}

\begin{opomba}
\hfill
\begin{itemize}
\item Po konstrukciji je $\F_f$ odprta, $\mathcal{J}_f$ pa zaprta, lahko pa sta obe prazni. Primera za to sta:
\begin{itemize}
\item $f(z) = z + 1$ $\Longrightarrow$ $f^n(z) = z + n \longrightarrow \infty$ enakomerno na kompaktih na $\hat{\C}$. Torej je $\mathcal{J}_f = \emptyset$.
\item Lattesova funkcija $f(z) = \frac{(z^2 + 1)^2}{4z(z^2 - 1)}$ ima $\mathcal{J}_f = \hat{C}$ in $\F_f = \emptyset$.
\end{itemize}
\item Za polinome se definiciji $\mathcal{J}_f$ in $\F_f$ poenostavita, kar bomo spoznali kasneje.
\end{itemize}
\end{opomba}

\begin{izrek}
Množici $\mathcal{J}_f$ in $\F_f$ sta naprej in nazaj invariantni tj. $f(\mathcal{J}_f) = f^{-1}(\mathcal{J}_f) = \mathcal{J}_f$ in $f(\mathcal{F}_f) = f^{-1}(\mathcal{F}_f) = \mathcal{F}_f$.
\end{izrek}

\begin{proof}[Dokaz:]
Dovolj je dokazati zvezo le za $\F_f$. Naj bo $z\in U$, kjer je $U \subset D$, na kateri so iterati normalni. Potem sta tudi množici $f^{-1}(U)$ in $f(U)$ odprti, seveda pa je družina $\{f^n \,|\, n\in \N\}$ normalna tudi na njih. Od tod sledi:
$$
f^{-1}(\F_f) \subseteq \F_f \text{ in } f(\F_f) \subseteq \F_f.
$$
Na prvo relacijo dodamo $f$ in dobimo:
$$
\F \subseteq f(\F_f) \text{ oz. } \F = f(\F).
$$
Posledično pa je tudi $\F_f \subseteq f^{-1}(\F_f)$ oz. $\F_f = f^{-1}(\F_f)$.
\end{proof}

Povezanim komponentam $\F_f$ pravimo Fatoujeve komponente in se slikajo ena v drugo.\\ 
Nek primerček.


\section{Napolnjena Juliajeva množica}

V tem razdelku se omejimo na ne linearne polinome
$$
p(z) = a_d z^d + \dots + a_0, \quad d \geq 2.
$$

Vsi taki polinomi imajo v $z = \infty$ superprivlačno točko. Zato lahko Juliajevo 
in Fatoujevo množico definiramo na alternativen način:
\begin{itemize}
    \item  
$$
\mathcal{K}_p \coloneqq \{z\in \C \,|\, |R^n(z)| \text{ omejena} \forall n\in \N\}
$$
napolnjena Juliajeva množica.
\item 
$$
\mathcal{J}_p = \partial \mathcal{K}_p \,\text{ in }\, \mathcal{F}_p = \hat{\C}\slash \mathcal{J}_p
$$
naši karakteristični množici. 
\end{itemize}
Ker je $\infty \in \F$ in $A_p(\infty) \subset \F_p$, se ta definicija 
ujema s standardnima. To pomenim, da je dinami blizu $z = \infty$ 
konjugirana dinamiki $z \mapsto z^d$ blizu $z = 0$.\\
Slikača klopotača\\
\begin{posledica}
Za polinome stopnje $d\geq 2$ je $\mathcal{J}_p$ vedno neprazna in omejena.
\end{posledica}

\begin{proof}[Dokaz:]
    Neprazna je, ker obstaja vsaj še ena fiksna točka, ki ni $z = \infty$
    torej je $\partial A_p(\infty) \neq \emptyset$. Ostalo sledi iz dejstva, 
    da je $A_p \subset \F_p$.
\end{proof}

Ugotovili smo, da za dovolj velike $z\in \C$ orbita pobegne proti $\infty$.
Številom, ki opredeljujejo zadostno velikost pravimo \textbf{radij pobega}.

\begin{izrek}[Radij pobega]
Naj bo $P$ polinom stopnje $d\geq 2$. Naj bo 
\begin{equation}
R \coloneqq \max\Big\{\frac{3}{a_d}, \frac{2}{a_d}(|a_0| + |a_1| + \dots + |a_{d-1}|)
, 1 \Big\}.
\end{equation}
Če je $|z| \geq R$, velja 
\begin{equation}
\lim_{n\rightarrow \infty} |p^n(z)| = \infty.
\end{equation}
\end{izrek}

\begin{proof}[Dokaz:]
Naj bo $|z| \geq R$. Naredimo oceno: 
\begin{align*}
|p(z)| &= |a_d z^d + \dots + a_z + a_0| = \\
&= |a_d z^d| \cdot \Big|1 + \frac{a_{d-1}}{a_d z} + \dots + \frac{a_1}{a_d z^{d-1} + \frac{a_0}{a_d z^d}}\Big| \\
&\leq \frac{|a_{d1}|}{|a_d|\cdot |z|} + \dots + \frac{|a_0|}{|a_d|\cdot |z|^d} \leq \\
&\leq \frac{|a_{d-1}| + \dots + |a_0|}{|a_d| \cdot |z|} \leq \frac{1}{2}
\end{align*}

\begin{align*}
|p(z)| &= |a_d| \cdot |z|^d \cdot |1 + s| \geq |a_d|\cdot |z|^d \cdot (1 - |s|)\\
&\geq |a_d| \cdot |z|^d \frac{1}{2} \geq \frac{3}{|z|} \cdot |z|^d \frac{1}{2} \geq
&\geq \frac{3}{2} |z| \geq \frac{3}{2} R.
\end{align*}
Induktivno ugotovimo:
$$
|P^n(z)| \geq \Big(\frac{3}{2} \Big)^n \cdot R \xrightarrow{n\rightarrow \infty} \infty
$$
\end{proof}

\begin{opomba}
Ta radij ni optimalen, lahko bi še prilagajali konstanti $3$ in $2$. 
Alternativna verzija pravi npr. da je dober tudi 
$$
R = \Big\{... \Big\}
$$
\end{opomba}

Posledica tega izreka je algoritem za približno risanje $\mathcal{K}_p$, ki 
ima naslednje korake:
\begin{itemize}
    \item Določi $R$ iz izberi $N>> 1$.
    \item Naključno izbiraj točke $z_0 \in \D(0, R)$, če za vse $n\in \{1, 2, \dots, n\}$
    velja, da je $|P^n(z_0)| < R$, jih obarvaj.
\end{itemize}

Dodatno lahko razmišljamo na sledeč način. Točka $z = \infty$ je edina 
fiksna točka v komponenti $A_p(\infty)$. Torej je poleg $\partial A_p(\infty)$
edina možna limita zaporedja v $A_p(\infty).$ Izberemo $z_0 \in A_p(\infty) \slash \{\infty\}$
in tvorimo zaporedje $z_n \in P^{-n}(z_0)$ (jemljemo praslike). To zaporedje 
ne more konvergirati k $\infty$, saj gre tja zaporedje $P^n(z_0)$. Torej 
konvergira k $\mathcal{J}_P = \partial A_p(\infty)$. To pa nam da še en 
možen algoritem za približno risanje $\mathcal{J}_p$, ki mu pravimo \textbf{vzvratna iteracija}: 
\begin{itemize}
    \item Določi $R$ in izberi $N >> 1$. 
    \item Naključno izbiraj $z_0 \in \partial \D(0, R)$. Nariši množico 
    $P^{-N}(z_0)$ (kar vse praslike). 
\end{itemize}
Naš modelni primer $p(z) = z^2$ in $R = \max\Big\{ \frac{1}{|1|}, \frac{2}{|1|}\cdot 0, 1\} = 3$. 
Primerčiča:\\

\begin{izrek}[Izrek o dihotomiji]
Juliajeva množica nelinearnega polinoma je povezanan natanko tedaj, ko so vse 
kritične točke v $\mathcal{K}_p$.
\end{izrek}

\begin{opomba}
$c\in \C$ je kritična točka za $p$, če je $p'(c) = 0$. To pomeni, da je 
razvoj okoli $c$ enak:
$$
p(z) = a_0 + a_k(z-c)^k + a_{k+1}z^{k+1} + z^d a_d, \quad k \geq 1.
$$
To pomeni, da lokalno oz. za vrednosti blizu $z = c$ funkcija $p$ 
"deluje kot" $z \mapsto z^k$ v posebnem, na okolici take točke $p$ ni
injektivna.
\end{opomba}

\begin{primer}
Oglejmo si primer $z \mapsto z^2$ in tri tipe krožnic ter njihovih praslik: 
kul slikice\\
Morala: če krožnica trči v kritično vrednost, je njena praslika topološko 
ekvivalentna osmici, krožnice v njej pa imajo dve nepovezani komponenti v 
prasliki.
\end{primer}

\begin{proof}[Dokaz(ideja):]
Ločimo dva primera:
\begin{enumerate}
    \item v $A_p(\infty)$ ni kritičnih točk in posledično tudi kritičnih vrednosti ne. 
    Za poljuben $r \geq R$, kjer je $R$ radij pobega, je množica $P^{-n}(\partial \D(0, r))$, 
    $n\in \N$, sklenjena krivulja brez samopresečišč (krožnica v topološkem smislu).
    V limiti pa dobimo $\mathcal{J}_p$, ki je povezana. 
    \item Če $A_p(\infty)$ vsebuje kritično točko in posledično tudi kritično vrednost 
    , potem bo praslika križnice $D(0, |p(c)|)$ unija več topoloških krožnic s 
    skupnim presečiščem. Manjše krožnice pa bodo razpadle na več komponent. \\
    Neki lepega
\end{enumerate}
\end{proof}

\section{Drugi Montelov izrek}

V tem razdelku bomo spoznali enega bolj pomembnih izrekov v kompleksni analizi, 
ki močno opredeli lastnosti množice $\mathcal{J}_R$. Omejili se bomo 
na racionalne funkcije stopnje $d \geq 2$. 

\begin{izrek}[Drugi Montelov izrek]
Če obstajo vrednosti $a, b, c \in \hat{\C}$, ki so različne. ter za družino 
$\F \{R: D \subseteq \hat{\C} \rightarrow \hat{\C}$ \,|\, R \text{ racionalna}\} 
velja $\{a, b, c\} \cap R(D) = \emptyset$ za vsak $R \in \F$, potem je 
$\F$ normalna. 
\end{izrek}

\begin{opomba}
\begin{itemize}
    \item Izrek velja tudi za družine meromorfnih funckij na $D \subset \C$ tj. 
    za funkcije brez bistvenih singularnosti.
    \item Izrek je pogosto podan za družine holomorfnih funkcij $\F \subset \O(D)$, 
    za $D \subset \C$, takrat je dovolj, da izpusti dve točki $a, b \in \C$, saj 
    za $c$ vzamemo $\infty$ (elementi $\F$ nimajo polov).
\end{itemize}
\end{opomba}

\begin{proof}[Dokaz(ideja).]
Kjučen del dokaza je obstoj surjektivne holomorfne preslikave 
$$
h: \D \rightarrow \hat{\C}\backslash \{0, 1, \infty\},
$$
ki nima kritičnih točk oz. je krovna projekcija tj. zožitev $h$ na vsako 
komponentno take praslike je biholomorfna. Od tu dalje brez škode za splošnost 
predpostavimo, da so $a = 0$, $b=1$ in $c = \infty$. Če to ni res, vse 
$R \in \F$ konjugiramo z ustrezno Möbiousovo preslikavo. 

Sedaj je preslikave "dvignemo na krov"
\end{proof}



\end{document}